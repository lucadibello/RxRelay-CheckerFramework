\documentclass[a4paper, 11pt]{article}
\usepackage[top=3cm, bottom=3cm, left = 2cm, right = 2cm]{geometry}
\geometry{a4paper}
\usepackage[utf8]{inputenc}
\usepackage{textcomp}
\usepackage{graphicx}
\usepackage{amsmath,amssymb}
\usepackage{bm}
\usepackage[pdftex,bookmarks,colorlinks,breaklinks]{hyperref}
\hypersetup{linkcolor=black,citecolor=black,filecolor=black,urlcolor=black} % black links, for printed output
\usepackage{memhfixc}
\usepackage{pdfsync}
\usepackage{fancyhdr}
\pagestyle{fancy}
\setlength{\headheight}{13.6pt}

\title{Extented Java Typechecking with Checker Framework \\[1ex] \large Software Analysis - Assignment 2}
\author{Luca Di Bello}
\date{\today}

\begin{document}
\maketitle
\tableofcontents

\section{Introduction}

This assignment aims to extend the Java type system with the \textit{Checker Framework}\footnote{\url{https://checkerframework.org/}}. The \textit{Checker Framework} is a powerful tool that integrates with the Java compiler to detect bugs and verify their absence at compile time. This is done by pluggable type-checkers, which via explicit annotations in the code, are able to check for a wide range of errors, such as null pointer dereferences, type casts, and array bounds.  It includes over 20 type-checkers, which can be used to verify a wide range of properties. Some of the most useful type-checkers include:

\begin{itemize}
    \item \textbf{Nullness Checker}: Prevent any \texttt{NullPointerException} by ensuring that variables are not null when dereferenced.
    \item \textbf{Index Checker}: Prevents array index out-of-bounds errors by ensuring that array accesses are always within bounds.
    \item \textbf{Regex Checker}: Prevents runtime exceptions due to invalid regular expressions by checking the syntax of regular expressions at compile time.
\end{itemize}

\textit{Note: a full list of available checkers can be found on the Checker Framework manual in the \texttt{Introduction} section}

\pagebreak

\noindent To showcase the effectiveness of this tool, the Checker Framework's index checker has been integrated into an existing codebase to prevent any out-of-bounds access to arrays. The project is a legacy Java library named \texttt{RxRelay}\footnote{\url{https://github.com/JakeWharton/RxRelay}}, a small library that aims to extend the capabilities of a famous Java library called \texttt{RxJava} by providing a set of \texttt{Relay} classes that act as both an \texttt{Observable} and a \texttt{Consumer}. The library is used to relay events from one component to another, and it is widely used in Android applications. \cite{rxrelay:readme}

The library is composed of a single package, \texttt{com.jakewharton.rxrelay2}, and it contains a total of 5 classes. The library is built using Gradle, and it is compatible with Java 8 and above.

\pagebreak

\section{Project setup}


includes pluggable compiler plug-ins that can find bugs or verify their absence.

\pagebreak

\section{System Design}

\pagebreak

\section{Evaluation}

\pagebreak

\section{Conclusions and Future Work}

\nocite{*} % Force to show all references in references.bib file
\bibliographystyle{abbrv}
\bibliography{references}  % need to put bibtex references in references.bib
\end{document}
